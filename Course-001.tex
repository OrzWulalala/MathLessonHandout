\documentclass[UTF8]{article}

\usepackage{CTEX}
\begin{document}  

\section{集合初步}

\subsection{集合的基本概念:}

集合:指具有某种特定性质的具体的或抽象的对象汇总而成的集体。我们将构成集合的这些对象称为该集合的元素。

\subsubsection{Example I:}
$S = \{1, 2, 3\}$ 表示S是由1, 2, 3三个数字构成的集合\\

全体自然数构成的集合 $N = \{0, 1, 2, 3, ...\}$\\

全体整数构成的集合 $Z = \{-\infty ...  -1, 0, 1,  ... +\infty \}$ \\

全体有理数构成的集合 $Q$\\

全体实数构成的集合 $R$\\

通常使用大写字母$S, A, B$等表示一个集合,使用小写字母表示集合中的元素。若$x$是集合$S$中的元素,称$x$属于$S$,记作$x \in S$。若$y$不是集合S中的元素,称$y$不属于$S$,记作$y \notin S$。


在集合中,任意两个元素都不相同,即每个元素只能出现一次。

\subsection{集合间的关系:}

\subsubsection{子集:}

设$S, T$是两个集合,如果集合$S$中的所有元素都属于集合$T(\forall x \in S, x \in T)$,称$S$是$T$的子集,记作$S\subseteq T$;特别的,如果存在集合$T$中的元素$x$,且$x$不属于$S(\exists x \in T, x\notin S)$,我们称$S$是$T$的真子集,记作$S \subset T$。

\subsubsection{空集:}

空集是一类特殊的集合,空集内不包含任何元素,例如$\{x|x^2 + 1 = 0, x \in N \}$就是一个空集。我们认为空集是任何一个集合的子集,同时也是任意非空集合的真子集。

\subsubsection{集合的交与并:}

对于集合$A$与集合$B$\\

定义$A$与$B$的交集为:$A \cap B = \{x | x\in A \quad and \quad x\in B\}$\\
 
定义$A$与$B$的并集为:$A \cup B = \{x | x\in A \quad or \quad x\in B\}$

\subsubsection{全集与补集:}

一般的,如果一个集合中包含我们研究问题中涉及的所有元素,我们称这个集合为全集,通常记作$U$。\\

对于集合$U$的子集$A$,定义集合$A$相对于全集$U$的补集:$A^{'} = \{x|x\in U \quad and \quad x\notin A\}$。

\subsection{集合运算定律:}

\subsubsection{交换律:}
$$A \cap B = B \cap A, A\cup B = B\cup A$$

\subsubsection{结合律:}
$$A\cup(B\cup C) = (A\cup B)\cup C$$
$$A\cap(B\cap C) = (A\cap B)\cap C$$

\subsubsection{分配对偶律:}
$$A\cap(B\cup C) = (A\cap B) \cup (A\cap C)$$
$$A\cup(B\cap C) = (A\cup B) \cap (A\cup C)$$

\subsection{容斥原理:}

对于一个集合$A$,我们称其中的元素个数为集合的基数,记作$card(A)$。\\

显然,有如下的等式:
$$card(A \cup B) = card(A) + card(B) - card(A \cap B)$$
$$card(A \cup B \cup C) = card(A) + card(B) + card(C) $$
$$- card(A \cap B) - card(B \cap C) - card(C \cap A) + card(A \cap B \cap C)$$

\subsubsection{例题:试求1$\sim$600中能被2或3整除的数字个数}

定义集合$A = \{x|x\%2 = 0, 1 <= x <= 600\}, B = \{x|x\%3 = 0, 1 <= x <= 600\}$,显然$card(A) = 300, card(B) = 200$

又有$A \cap B = \{x|x\%6 = 0, 1 <= x <= 600\}, card(A\cap B) = 100$

于是$card(A\cup B) = card(A) + card(B) - card(A\cap B) = 400$,即为题目所求

\subsection{思考题:}

\subsubsection{$Question_1$:}

给定一个包含5个元素的集合$S = \{x_1, x_2, ... , x_5\}$,求集合$S$的真子集有多少个?

\subsubsection{$Question_2$:}

试验证$De\cdot Morgan$定律:
$$(A \cup B)^{'} = A^{'} \cap B^{'}$$
$$(A \cap B)^{'} = A^{'} \cup B^{'}$$

\section{函数与映射:}

\subsection{映射的定义:}

两个非空集合$X, Y$中存在对应关系$f$,而且对于集合$X$中的每一个元素$x$,$Y$中总有唯一的一个元素$y$与其对应,我们称这种对应关系为集合$X$到集合$Y$的映射,记作$f: X \mapsto Y$,其中$y$称作$x$在映射$f$下的像,记作:$y = f(x)$。$x$称为$y$关于映射$f$的原像。集合$X$中所有元素在$f$作用下的像的集合称为$f$的值域,记作$f(X)$\\

对于从集合$X$到集合$Y$的映射$f: X \mapsto Y$,我们有:

\subsubsection{满射:}

$$\forall y \in Y, \exists x \in X, s.t. f(x) = y$$


\subsubsection{单射:}

$$\forall x_1, x_2 \in X (x_1\neq x_2), f(x_1) \neq f(x_2)$$

\subsubsection{双射(一一映射):}

映射$f$既是满射,又是单射时,称映射$f$是一个双射。

\subsubsection{一些例子:}

令集合$X$为全体自然数的集合,$Y = \{0, 1\}$,定义映射$$f : X\mapsto Y, f(x) = x\%2$$

我们称映射$f$是$X\to Y$的一个满射。\\
\\

令集合$X = \{1, 2, 3, 4, 5\}$, 集合$Y$是全体有理数的集合,定义映射$$f:X\mapsto Y, f(x) = \frac{x}{6}$$

我们称映射$f$是$X\to Y$的一个单射。\\
\\

令集合$X = \{1, 2, 3, 4, 5\}$, 集合$Y = \{3, 5, 7, 9, 11\}$, 定义映射$$f:X\mapsto Y, f(x) = 2x + 1$$

我们称映射$f$是$X\to Y$的一个双射。\\


\subsection{函数:}

假设$X, Y$是非空的数集,如果按照某种确定的对应关系$f$,使得对于集合$X$中的任意一个数x,在集合$Y$中都有唯一确定的数与之对应,那么就称映射$f:X\mapsto Y$是从集合$X$ 到集合$Y$的一个函数,记作$y = f(x), x \in X$.\\

在函数$x$称作自变量,$y$称作$x$的函数;集合$X$称作函数的定义域,与$x$对应的$y$叫做函数值(因变量),所有函数值构成的集合$\{f(x)|x\in X\}$称作函数的值域,$f$叫做对应法则。定义域,值域以及对应法则称为函数的三要素。

\subsubsection{函数图像:}

设函数$y = f(x),x\in D$,在平面直角坐标系中,我们将有序数对$(x, f(x))$视作一个平面上的点,那么点集$\{(x, f(x)) | x \in D\}$所构成的图形便称为函数$y = f(x),x\in D$的函数图像。



\subsubsection{基本的函数性质:}

对于函数$y = f(x), x\in D$,有如下几种基本特性:

单调性:对于$D$上的一个子集$I$, 当且仅当:\\
$\forall x_1, x_2 \in I, x_1 < x_2 \rightarrow f(x_1) < f(x_2)$,我们称函数$f$在$I$上单调递增;\\
$\forall x_1, x_2 \in I, x_1 < x_2 \rightarrow f(x_1) > f(x_2)$,我们称函数$f$在$I$上单调递减。\\

奇偶性:对于$\forall x \in D, f(x) = f(-x)$,称函数$y = f(x)$为偶函数;对于$\forall x \in D, -f(x) = f(-x)$,称函数$y = f(x)$为奇函数。\\

周期性:如果$\exists T, T > 0, s.t. \forall x \in D, f(x) = f(x + T)$,则称函数$f$是一个周期函数,$T$称作函数$f$的周期(通常指最小正周期)

\subsubsection{一些常见的初等函数:}

试求出下列初等函数的值域并绘出函数图像,分析其函数特性:

一次函数:$f(x) : y = kx + b, k, b\in Z^{+}, x \in R$\\

反比例函数:$f(x) : y = \frac{k}{x}, k\in R, x \in (-\infty, 0)\cup (0, \infty)$\\

二次函数:$f(x) : y = ax^2 + bx + c, x\in R$\\

指数函数:$f(x) : y = a^x, a > 0, x \in R$\\

对数函数:$f(x) : y = log_a x, a > 0, x \in R^{+}$\\

正弦函数:$f(x) : y = sin(x), x \in R$\\

余弦函数:$f(x) : y = cos(x), x \in R$

\end{document}
