\documentclass[UTF8]{article}
\usepackage{graphicx}
\usepackage{ctex}
\begin{document}

\section{Review:}

\subsection{Question I:}

给定函数$f(x) : y = |x|, x \in R, g(x) : y = x^2 + 3x - 4, x \in R$,求两函数交点的坐标

\subsection{Question II:}

试证明$f(x) : y = x^2 + 5x + 3, x \in R$与$g(x) : y = 3x + 2, x \in R$有且仅有一个交点,并求出交点坐标\\

\section{导数与切线方程:}

设函数$y = f(x), x\in D$在定义域上某个邻域$\delta \to 0, (x_0 - \delta, x_0 + \delta)$内可导,定义函数$f(x)$在$x = x_0$处的导数$f^{'}(x) = \lim_{\Delta x \to 0}\frac{f(x + \Delta x) - f(x)}{\Delta x}$

我们不纠结于函数可导本身的定义,先考虑导数本身的含义:对于函数$y = f(x)$在$x = x_0$处的导数$f^{'}(x_0)$, 实际上代表着函数$f(x)$在$x = x_0$处的变化率。

\subsection{简单函数的求导:}

\begin{itemize}
	\item 我们假设$s$表示路程,$v, t$分别表示速度与时间,我们将路程$s$视作时间$t$的函数$s = u(t) = v\cdot t$,容易得出在$t = t_0$处函数$u$关于时间$t$的导数为:

	$$u^{'}(t_0) = \lim_{\Delta t \to 0}\frac{u(t_0 + \Delta t) - u(t_0)}{\Delta t}$$ 
	$$= \lim_{\Delta t \to 0}\frac{v\cdot (t_0 + \Delta t) - v\cdot t_0}{\Delta t}$$
	$$= \lim_{\Delta t \to 0}\frac{v \cdot \Delta t}{\Delta t} = v$$

	显然距离$s$关于时间$t$的变化率就是实时速度$v$;\\

	\item 设有一次函数$y = f(x) = k\cdot x + b, x\in R$.易得$f(x)$在$\forall x_0\in R$处的导数值为$f(x_0) = k$.形式化的看,对于一次函数来说,函数的导数值为常数,也可以理解为函数图象的斜率。

	\item 设函数$y = f(x) = x^2, x\in R$,对于$\forall x_0 \in R$, 函数在$x = x_0$处的导数为:
	$$f^{'}(x_0) = \lim_{\Delta x \to 0} \frac{f(x_0 + \Delta x) - f(x_0)}{\Delta x}$$
	$$= \lim_{\Delta x \to 0} \frac{{x_0}^2 + 2x_0 \Delta x + \Delta x^2 - {x_0}^2}{\Delta x}$$
	$$= \lim_{\Delta x \to 0} 2x_0 + \Delta x = 2x_0$$

	对于在定义域上处处可导的函数$f(x)$,我们记$f^{'}(x)$为原函数的导函数;对于$f(x) = x^2$,易得导函数$f^{'}(x) = 2x$.
\end{itemize}

\subsection{常见初等函数的导数:}

\begin{itemize}
	\item $f(x) = C$, $f^{'}(x) = 0$
	\item $f(x) = kx + b$, $f^{'}(x) = k$
	\item $f(x) = x^n$, $f^{'}(x) = n\cdot x^{n-1}$
	\item $f(x) = a^x$, $f^{'}(x) = a^x\cdot ln a$
	\item $f(x) = sinx$, $f^{'}(x) = cosx$
	\item $f(x) = cosx$, $f^{'}(x) = sinx$
\end{itemize}

\subsection{求导法则:}
设有函数$f(x), g(x), x\in D$,下简称函数$f, g$
\begin{itemize}
	\item $\forall k \in R, (k\cdot f)^{'} = k \cdot f^{'}$
	\item $(f + g)^{'} = f^{'} + g^{'}$
	\item 更普遍的,对于$\forall \alpha, \beta \in R, (\alpha\cdot f + \beta\cdot g)^{'} = \alpha\cdot f^{'} + \beta\cdot g^{'}$
	\item $(f\cdot g)^{'} = f\cdot g^{'} + f^{'}\cdot g$
\end{itemize}

\subsection{Practice:}

\begin{itemize}
	\item $(ax^2 + bx + c)^{'} = $
	\item $(\frac{1}{x})^{'} = $
	\item $(\frac{x^2 + 1}{2x})^{'} = $
\end{itemize}


\subsection{切线方程:}

对于函数$f(x), x\in D$,如果在$x = x_0$处可导,那么$f^{'}(x_0)$就是过点$(x_0, f(x_0))$处切线的斜率。过该点且斜率为$f^{'}(x_0)$的直线即为函数$f(x)$在$x = x_0$处的切线方程。

\begin{itemize}
	\item 设$y = f(x) = 3x^2 + 5x + 2$, 求点$(1, f(1))$处的切线方程
	\item 设$y = f(x) = x + \frac{1}{x}$, 求点$(-2, f(-2))$处的切线方程
\end{itemize}
\section{复合函数:}

设$y$是$u$的函数$y = f(u)$, $u$是$x$的函数$u = g(x)$,如果$g(x)$的值全部或部分落在$f(x)$的定义域内,则$y$通过$u$成为$x$的函数,记作$y = f(g(x))$,称为由函数$y = f(u)$与$u = g(x)$复合而成的函数。

\subsection{一些例子:}

\begin{itemize}
	\item $y = f(u) = \sqrt{u - 1}, u = g(x) = x^2 - 3$,得到$y$通过$u$得到关于$x$的复合函数$y = f(x) = \sqrt{x^2 - 4}$,其中$x$的定义域为$(-\infty, -2] \cup [2, +\infty)$,函数的值域为$R^{+}$

	\item $y = f(u) = 3u^2 - 1, u = g(x) = -x + 2$,得到$y$通过$u$得到关于$x$的复合函数$y = f(x) = 3x^2 - 12x + 11$,其中$x$的定义域为$R$, 函数的值域为$[-1, +\infty)$
\end{itemize}



\subsection{Practice:}

\begin{itemize}
	\item 令$y = f(u) = e^u - 1, u \in R, u = g(x) = x^2 + 2x - 2, x \in R$,求$y$关于$x$的复合函数$y = f(g(x))$的显示方程,并求出该函数的定义域与值域.

	\item 令$y = f(u) = \frac{1}{u}, u \in (-\infty, 0)\cup (0, +\infty), u = g(x) = log_2 x + 2, x \in (0, +\infty)$,求$y$关于$x$的复合函数$y = f(g(x))$的显示方程,并求出该函数的定义域与值域.
\end{itemize}



\section{向量:}

数学中,向量一般指具有大小与方向的量,通常记作$\vec{u}, \vec{x}, ... $;一般的,我们可以在平面直角坐标系$XOY$中,将一个向量表示为从原点$O(0, 0)$指向点$A(x_A, y_A)$的箭头,记作向量
$
\overrightarrow{OA} = 
\left(
\begin{array}{l}
x_A\\
y_A
\end{array}
\right)
$,更普遍的,从点$A(x_A, y_A)$指向点$B(x_B, y_B)$的向量记为
$
\overrightarrow{AB} = 
\left(
\begin{array}{l}
x_B - x_A\\
y_B - y_A
\end{array}
\right)
$


\subsection{相关概念:}

\begin{itemize}
	\item 向量的模长:即为向量的长度,通常在二维平面内的向量$\vec{u} = (x, y)$,将其模长记为$|\vec{u}| = \sqrt{x^2 + y^2}$。需要注意的是向量与向量之间不能直接比较大小,不存在$\vec{u} > \vec{v}$的概念.
	\item 共线向量:若向量$\vec{u}, \vec{v}$具有相同的方向,则称两个向量是共线的.
	\item 零向量:模长为0的向量;特殊的,我们认为零向量没有方向,或者说零向量与任意向量方向相同(共线).
	
\end{itemize}


\subsection{向量的运算:}

\begin{itemize}
	\item 向量的数乘:对于向量$\vec{u} = \left(
\begin{array}{l}
x_u\\
y_u
\end{array}
\right), \alpha \in R$,有$\alpha\cdot\vec{u} = \left(
\begin{array}{l}
\alpha\cdot x_u\\
\alpha\cdot y_u
\end{array}
\right)$,通常我们将向量的数乘理解为对该向量的伸缩变换,即不改变向量的方向,只改变其长度的变换,对于$\forall \alpha\in R, |\alpha\cdot \vec{u}| = |\alpha|\cdot |\vec{u}|$

	\item 向量加法:对于向量
$\vec{u}= 
\left(
\begin{array}{l}
x_u\\
y_u
\end{array}
\right), 
\vec{v}= 
\left(
\begin{array}{l}
x_v\\
y_v
\end{array}
\right)$, 有
$\vec{u} + \vec{v} = 
\left(
\begin{array}{l}
x_u + x_v\\
y_u + y_v
\end{array}
\right)$;显然,对于向量加法与向量的数乘,满足交换律$\vec{u} + \vec{v} = \vec{v} + \vec{u}$与结合律:$\forall \alpha \in R, \alpha \cdot (\vec{u}+ \vec{v}) = \alpha\cdot\vec{u} + \alpha\cdot\vec{v}$;更普遍的,对于$\alpha, \beta \in R$,我们有$\alpha \cdot \vec{u} + \beta \cdot \vec{v} = 
\left(
\begin{array}{l}
\alpha x_u + \beta x_v\\
\alpha y_u + \beta y_v
\end{array}
\right)$,事实上,向量间的加法运算相当于两个向量所代表运动的叠加。

	\item 向量的点积:对于二维平面中的向量$\vec{u} = \left(
\begin{array}{l}
x_u\\
y_u
\end{array}
\right), \vec{v} = \left(
\begin{array}{l}
x_v\\
y_v
\end{array}
\right)$,定义点积运算$\vec{u}\cdot\vec{v} = x_u\cdot x_v + y_u\cdot y_v$.

事实上,向量$\vec{u}, \vec{v}$的点积运算相当于计算$\vec{u}$在向量$\vec{v}$上的投影的长度与向量$\vec{v}$的模长的乘积,即$\vec{u}\cdot\vec{v} = |\vec{u}|\cdot|\vec{v}|\cdot sin<\vec{u}, \vec{v}>$
\end{itemize}


\subsection{一些例子:}
\begin{itemize}
	\item 对于平面上的点$A(1, 2)$,设向量$\vec{a} = \overrightarrow{OA} = 
\left(
\begin{array}{l}
1\\
2
\end{array}
\right)$,则$2\cdot \vec{a} = 
\left(
\begin{array}{l}
2\\
4
\end{array}
\right)$,相当于将向量$\vec{a}$延长了1倍的长度;另外,$-1\cdot\vec{a} = -\vec{a}$称做向量$\vec{a}$的相反向量,模长与$\vec{a}$相同,方向相反。

	\item 对于平面上的点$A(3, 1), B(0, 2)$,设向量$\vec{a} = \overrightarrow{OA} = 
\left(
\begin{array}{l}
3\\
1
\end{array}
\right), \vec{b} = \overrightarrow{OB} = 
\left(
\begin{array}{l}
0\\
2
\end{array}
\right), \vec{a} + \vec{b} = 
\left(
\begin{array}{l}
3 + 0\\
1 + 2
\end{array}
\right) = 
\left(
\begin{array}{l}
3\\
3
\end{array}
\right)$;

\begin{figure}{}
\centering\includegraphics[scale = 0.25]{vec_add.jpeg}
\end{figure}
\end{itemize}




\begin{itemize}
	\item 设$XOY$平面中的向量$$\vec{a} = 
\left(
\begin{array}{l}
5\\
2
\end{array}
\right),
\vec{b} = 
\left(
\begin{array}{l}
1\\
4
\end{array}
\right)\to \vec{a}\cdot\vec{b} = 5\cdot 1 + 2\cdot 4 = 13
$$

$$\vec{a} = 
\left(
\begin{array}{l}
2\\
5
\end{array}
\right),
\vec{b} = 
\left(
\begin{array}{l}
3\\
-3
\end{array}
\right)\to \vec{a}\cdot\vec{b} = 2\cdot 3 + 5\cdot (-3) = -9$$

$$\vec{a} = 
\left(
\begin{array}{l}
3\\
-4
\end{array}
\right),
\vec{b} = 
\left(
\begin{array}{l}
12\\
9
\end{array}
\right)\to \vec{a}\cdot\vec{b} = 3\cdot 12 + (-4)\cdot 9 = 0$$

更一般的,对于向量$\vec{a}, \vec{b}$,令$\theta$表示$\vec{a}, \vec{b}$间的夹角,我们有:

$$
\left\{
\begin{array}{l}
\vec{a}\cdot\vec{b} > 0 \Rightarrow \theta < 90^{\circ}\\
\vec{a}\cdot\vec{b} = 0 \Rightarrow \theta = 90^{\circ}\\
\vec{a}\cdot\vec{b} < 0 \Rightarrow \theta > 90^{\circ}\\
\end{array}
\right.
$$
\end{itemize}

\subsection{BrainStorm:}

\begin{itemize}
	\item 设向量
$\vec{u} = 
\left(
\begin{array}{l}
1\\
2
\end{array}
\right), \vec{v} = 
\left(
\begin{array}{l}
3\\
1
\end{array}
\right)$,若向量$\vec{x} = \left(
\begin{array}{l}
6\\
7
\end{array}
\right)$可以表示为$\vec{x} = \alpha\cdot\vec{u} + \beta\cdot\vec{v}$,试确定$\alpha, \beta$的值.
	\item 考虑$XOY$平面上不重合的两点$A(x_A, y_A), B(x_B, y_B)$,记$\vec{a} = \overrightarrow{OA}, \vec{b} = \overrightarrow{OB}$,试证明对于$\forall  r \in [0, 1]$,向量$\vec{c} = r \cdot \vec{a} + (1 - r)\cdot \vec{b}$的终点$C$在线段$AB$上。
	\item 设直线方程$l: Ax + By + C = 0$,求该直线的法向量(垂直于该直线的向量) 。
	
\end{itemize}



\end{document}