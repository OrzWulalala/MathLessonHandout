\documentclass[UTF8]{article}

\usepackage{CTEX}
\begin{document} 

\section{线性空间与基向量:}
\subsection{线性空间:}

设有空间$V$,如果在$V$中的向量$\vec{x}, \vec{y}$,有$\forall \vec{x}, \vec{y} \in V,$且满足:
$$\forall k \in R, k\cdot \vec{x} \in V$$
$$\forall \alpha,\beta \in R, \alpha\cdot \vec{x} + \beta\cdot\vec{y} \in V$$
我们称空间$V$是线性空间。常见的$R, R^2, R^3$都是线性空间。

下面我们以二维平面$R^2$为例讨论有关线性空间的内容。

\subsection{线性空间的基} 
在二维平面$R^2$中,建立平面直角坐标系$XOY$,在这个坐标系中,有两条向量是非常特殊的,记为$\vec{e}_i = 
\left(
\begin{array}{l}
1 \\ 
0
\end{array}
\right),
\vec{e}_j = 
\left(
\begin{array}{l}
0 \\ 
1
\end{array}
\right)$,通常称这两条向量称为二维平面空间中的一对基向量;
$XOY$坐标系中任意一条向量$\vec{u} = 
\left(
\begin{array}{l}
x \\ 
y
\end{array}
\right)
$都可以看作$\vec{e}_i, \vec{e}_j$的线性组合:
$$
\left(
\begin{array}{l}
x \\ 
y
\end{array}
\right) = x\cdot
\left(
\begin{array}{l}
1 \\ 
0
\end{array}
\right) + y\cdot
\left(
\begin{array}{l}
0 \\ 
1
\end{array}
\right)
$$
相对应的,我们称平面$XOY$是由向量$\vec{e}_i, \vec{e}_j$张成的空间,记$
\left(
\begin{array}{l}
x \\ 
y
\end{array}
\right)
$为向量$\vec{u}$在这对基向量张成空间下的坐标。

默认状态下,我们认为一个向量的坐标即为它在$XOY$坐标系中的坐标。

\subsubsection{线性相关与线性无关:}
对于一组向量(多于一个)$\vec{a}_1, \vec{a}_2, ..., \vec{a}_n$,如果存在不全为$0$的系数$k_1, k_2, ..., k_n$,使得$\sum_{i = 1}^{k}{k_i\cdot \vec{a}_i} = 0$,则称$\vec{a}_1, \vec{a}_2, ..., \vec{a}_n$是一组线性相关的向量;

反之,若当且仅当$k_1 = k_2 = ... = k_n = 0$时,才能使$\sum_{i = 1}^{k}{k_i\cdot \vec{a}_i} = 0$成立,则称$\vec{a}_1, \vec{a}_2, ..., \vec{a}_n$是一组线性无关的向量。

\begin{itemize}
	\item 在$XOY$平面内,两个向量线性相关意味着两个向量是共线的。
	\item 在$XOY$平面内,超过三个向量一定线性相关。
\end{itemize}

我们说任意一组线性无关的向量可以张成一个线性空间,向量组中向量的个数称为该线性空间的秩。

\subsubsection{线性空间的基与向量坐标的计算:}

\begin{itemize}
	\item 设$XOY$中有向量$\vec{u} = 
\left(
\begin{array}{l}
5 \\ 
7
\end{array}
\right)
$,显然在以$\vec{e}_i = 
\left(
\begin{array}{l}
1 \\ 
0
\end{array}
\right), \vec{e}_j = 
\left(
\begin{array}{l}
0 \\ 
1
\end{array}
\right)$的原坐标系中坐标为$
\left(
\begin{array}{l}
5 \\ 
7
\end{array}
\right)$.

我们现在以另一组向量$\vec{e}_p = 
\left(
\begin{array}{l}
1 \\ 
1
\end{array}
\right), \vec{e}_q = 
\left(
\begin{array}{l}
-1 \\ 
1
\end{array}
\right)$为基张成另一个线性空间$V_1$,在这个新的线性空间中,向量$\vec{u}$的坐标就换成了另一对$
\left(
\begin{array}{l}
u_x \\ 
u_y
\end{array}
\right)
$,且满足$u_x\cdot \vec{e}_0 + u_y\cdot \vec{e}_1 = \vec{u}
$,即$$
u_x\cdot \left(
\begin{array}{l}
1 \\ 
1
\end{array}
\right) + u_y\cdot \left(
\begin{array}{l}
-1 \\ 
1
\end{array}
\right) = \left(
\begin{array}{l}
5 \\ 
7
\end{array}
\right)
$$

联立得到方程$$
\left\{
\begin{array}{l}
u_x - u_y = 5 \\ 
u_x + u_y = 7
\end{array}
\right.
$$

解方程得$$
\left\{
\begin{array}{l}
u_x = 6 \\
u_y = 1
\end{array}
\right.
$$

于是得到向量$\vec{u}$在线性空间$V_1$中的坐标为$
\left(
\begin{array}{l}
6 \\
1
\end{array}
\right)
$

	\item 
类似的,对于在空间$V_1$中的向量$\vec{u} = 
\left(
\begin{array}{l}
u_x \\ 
u_y
\end{array}
\right)
$,可以计算该向量在$XOY$平面中的坐标:
$$ \vec{e}_0 = 
\left(
\begin{array}{l}
1 \\ 
1
\end{array}
\right),
\vec{e}_1 = 
\left(
\begin{array}{l}
-1 \\ 
1
\end{array}
\right)$$

于是可以计算得出在原坐标系中:
$$\vec{u} = u_x\cdot \vec{e}_0 + u_y\cdot\vec{e}_1$$

即$$\vec{u} = u_x\cdot \left(
\begin{array}{l}
1 \\ 
1
\end{array}
\right) + u_y\cdot \left(
\begin{array}{l}
-1 \\ 
1
\end{array}
\right) = 
\left(
\begin{array}{l}
u_x - u_y \\ 
u_x + u_y
\end{array}
\right)$$
即为在$XOY$平面中的坐标

\end{itemize}

\section{线性变换:}



\end{document} 