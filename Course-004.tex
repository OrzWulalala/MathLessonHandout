\documentclass[UTF8]{article}
\usepackage{graphicx}
\usepackage{ctex}
\begin{document}

\section{数列:}

数列$a$可视作以正整数集$Z^{+}$为定义域的函数,形式上是一列有序的数;数列中每一个数都是该数列的一项,排在第一位的数为该数列的第$1$项,第二位的数为该数列的第$2$项,以此类推。第$n$位的数为该数列的第$n$项,记作$a_n$.

以下为几种常用的数列:

\subsection{等差数列:}

严格定义:$\{a_i|\forall i \in Z^{+}, a_{i+1} - a_i = d\}$,其中$d$是公差;

要描述一个需要等差数列,至少需要两个量:数列中某一项$a_i$,以及公差$d$.
\begin{itemize}
	\item 如已知数列首项$a_1$, 公差$d$, 易得数列通项公式$a_n = a_1 + (n-1)\cdot d$;
	\item 已知数列其中一项$a_n$以及公差$d$, 易得首项$a_1 = a_n - (n-1)\cdot d$, 再得出通项公式;
	\item 已知等差数列两项$a_i, a_j$, 可以直接得出公差$d = \frac{a_i - a_j}{i - j}$, 再计算得出首项和通项公式;
\end{itemize}

等差数列求和公式(高斯公式):$$a_i + a_{i+1} + a_{i+2} + ... + a_{j} = \frac{(a_i+a_j)\cdot (j-i+1)}{2}$$

记$$S = a_i + a_{i+1} + ... + a_{j}$$

同时有$$S = a_{j} + a_{j-1} + ... + a_i$$

对位相加得:$$2\cdot S = (a_i + a_j) + (a_{i+1} + a_{j-1}) + ... + (a_j + a_i)$$

由于数列$a$是等差数列,有$$a_i+a_j = a_{i+1} + a_{j-1} = ... = a_j + a_i$$

易得:$$2\cdot S = (j - i + 1)\cdot (a_i + a_j)$$

于是:$$S = \frac{(a_i+a_j)\cdot (j-i+1)}{2}$$

简记为首项加末项乘以项数除以二.

\subsection{等比数列:}

严格定义:$\{a_i|\forall i \in Z^{+}, \frac{a_{i+1}}{a_i} = q\}$,其中$q$为公比;

等比数列通项公式:$$a_n = a_1\cdot q^{n-1}, q\in R, q\neq 0$$

等比数列求和公式:$$a_i + a_{i+1} + ... + a_j = \frac{a_{j+1} - a_i}{q - 1}$$

设有等比数列:$a_n = a_1\cdot q^{n-1}$

记$$S = a_i + a_{i+1} + a_{i+2} + ... + a_j, i \leq j$$

等式两边同时乘以公差$q$,得到
$$q\cdot S = q\cdot a_i + q\cdot a_{i+1} + q\cdot a_{i+2} + ... + q\cdot a_j$$

$$q\cdot S = a_{i+1} + a_{i+2} + ... + a_j + a_{j+1}$$

下减上得到:
$$q\cdot S - S = a_{j+1} - a_i$$

于是有:$$S = \frac{a_{j+1} - a_i}{q - 1}$$


\subsection{Practice:}

\begin{itemize}
	\item 设有等差数列$a$,现已知$a_4 = 12, a_7 = 36$, 求该数列的首项,公差,以及通项公式.
	\item 设有等比数列$a$,现已知$a_4 = 64, a_6 = 128$, 求该数列的通项公式.
	\item 已知数列$a$中两项$a_8 = 27, a_{10} = -54$, 试验证该数列不可能是一个等比数列.
	\item 设数列$a_n = \frac{1}{3^n}$,$S_n = a_1 + a_2 + a_3 ... + a_n$,试求$\lim_{n\to \infty} S_n$的值.
	\item 设数列$a_n = q^n, q\in R, q\neq 0$,$S_n = a_1 + a_2 + a_3 ... + a_n$,试验证当$q < 1$时,$\lim_{n\to \infty} S_n = \frac{1}{\frac{1}{q} - 1}$
\end{itemize}

\newpage

\section{组合数学基础:}

\subsection{排列数:}

从$m$个物品中取出$n(n\leq m)$个物品,不同的取法个数(最终方案中取出的物品一样,但取出的顺序不同,视作不同的方案)记为排列数$A_m^n$.

计算式:$$A_m^n = \frac{m!}{(m-n)!} = m\cdot (m-1)\cdot (m-2) ... \cdot(m-n+1)$$

直观上可以理解为取$n$个物品,第一个物品有$m$种取法,第二个物品有$m-1$种取法,依此类推,第$n$个物品有$m-n+1$种取法;根据乘法原理,不同的取法有$m\cdot (m-1)\cdot (m-2) ... \cdot(m-n+1)$种.

\subsection{组合数:}

从$m$个物品中,任取$n(n\leq m)$个物品合并成一组,称为从$m$个不同物品中取出$n$个物品的一个组合;从$m$个不同物品中取出$n$个物品的所有组合的个数,称为从$m$个不同物品中取出$n$个物品的组合数.

计算式:$$C_m^n = \frac{A_m^n}{n!} = \frac{m!}{(m-n)!\cdot n!}$$

在上述组合数的计算式中,我们发现相较于排列数,组合数的方案中是不考虑取物品时的先后顺序的;同样是在$m$个不同物品中取$n$个,在所有的排列方案中,最终取得物品相同的每一种方案均会对应$n!$种不同的排列,这样的方案恰好对应一个组合,从而有$C_m^n = \frac{A_m^n}{n!}$,展开之后就是上述的计算式。


常用的组合恒等式:

$$C_m^n = C_m^{m-n}$$

显然,在$m$个物品中选$n$个的方案,相当于反选其余的$m-n$个物品,其方案数也相等.

$$C_m^n = C_{m-1}^{n-1} + C_{m-1}^n$$

对于$m$个物品中选$n$个物品的方案个数,由于不用考虑选择时得顺序,我们根据第$m$个物品是否在最终的方案中来对所有方案分类:
\begin{itemize}
	\item 如果第$m$个物品在最终的选择方案中,那么还需要在$1\to m-1$个物品中选择$n-1$个物品,有$C_{m-1}^{n-1}$种不同的方案.
	\item 如果第$m$个物品不在最终的选择方案中,那么相当于在$1\to m-1$个物品中选择$n$个物品,有$C_{m-1}^n$种不同的方案.
\end{itemize}

于是有上述的$C_m^n = C_{m-1}^{n-1} + C_{m-1}^n$.

\subsection{常见应用:}

\begin{itemize}
	\item 三枚奖牌,金银铜颁发给8个人中的3个人,有多少种方法? 换言之:8个人选出3个人进行排名,有多少种方法?

	\item 6个人两两分组,有多少种组队方式?

	\item 6个人分3组,各组人数分别为3人、2人、1人,共有多少分法?

	\item 6个人分成3组,各组人数分别为4人、1人、1人,共有多少种分法?

	\item $n$个信封,编号$1\to n$,现要将其分配到编号为$1\to n$的$n$个邮箱中,要求编号为$k$的信封不能放入编号为$k$的邮箱中,求一共有多少种分配方案.

	\item 有$n$个数字($1\to n$按顺序排列)和一个栈,每次可以按照顺序进栈一个数字或弹出栈顶的数字.求有多少种不同的出栈序列.(Catalan数)
\end{itemize}
\end{document}